\documentclass[12pt, letterpaper]{article}
\title{Assignment 4}
\author{Ming Li Liu}
\date{ \today }

\usepackage{amsmath,amssymb}

\begin{document}
\maketitle
\pagebreak

\begin{enumerate}
	\item 
	Let \(\vec{r}(t) = x(t) + y(t) + z(t)\) be some space curve describing the path of the particle.

	Its direction of motion at any given point is given by its first derivative:

	\[
	\frac{d \vec{r} }{dt} = 
	\frac{d \vec{r} }{dx} \frac{dx}{dt} +
	\frac{d \vec{r} }{dy} \frac{dy}{dt} +
	\frac{d \vec{r} }{dz} \frac{dz}{dt} 
	\]

	Given the level surface \(f(x, y, z) = 4 - x^2 - 2y^2 + 3z^2\), its gradient is:

	\[
		\nabla f(x, y, z) = \langle -2x, -4y, 6z \rangle
	\]

	

	Since the gradient is always perpendicular to the level surface at any point and that the direction of motion of the particle is perpendicular to the level surface as well, the direction of motion is therefore parallel to the gradient.
	

	

	
\end{enumerate}

\end{document}
